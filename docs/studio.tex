% !TEX encoding = UTF-8

%% Davide Imola - VR386238
%% Esame di LaTeX

\documentclass[a4paper,titlepage]{book}

\usepackage[italian]{babel}
\usepackage[nowrite,noadvisor]{frontespizio}
\usepackage{amsmath,amssymb,graphicx}
\usepackage{natbib}
\usepackage{url}

\begin{document}

\pagestyle{plain}

%Generazione Frontespizio
\begin{frontespizio}
\Universita{Verona}
\Dipartimento{Informatica}
\Corso[Laurea]{Informatica}
\Annoaccademico{2016--2017}
\Titoletto{Ingegneria del Software}
\Titolo{Documento dei Requisiti}
\Sottotitolo{Negozio CD musicali}
\Candidato[VR386253]{Andrea Slemer}
\Candidato[VR386238]{Davide Imola}
\Candidato[VR386665]{Giovanni Bellorio}
\end{frontespizio}

\tableofcontents
\clearpage
\thispagestyle{empty}

\chapter{Testo elaborato}
Si vuole progettare un sistema informativo per gestire le informazioni relative alla gestione di un negozio
virtuale di CD e DVD musicali (vende solo via web).
Il negozio mette in vendita CD di diversi generi: jazz, rock, classica, latin, folk, world-music, e cos\'i via.
Per ogni CD o DVD il sistema memorizza: un codice univoco, il titolo, i titoli di tutti i pezzi contenuti,
eventuali fotografie della copertina, il prezzo, la data dalla quale \'e presente sul sito web del negozio, il
musicista/band titolare, una descrizione, il genere del CD o DVD, i musicisti che vi suonano, con il
dettaglio degli strumenti musicali usati. Per ogni musicista il sistema registra il nome d’arte, il genere
principale, l’anno di nascita, se noto, gli strumenti che suona.
Sul sito web del negozio \'e illustrato il catalogo dei prodotti in vendita.
Cliccando sul nome del prodotto, appare una finestra con i dettagli del prodotto stesso.
I clienti possono acquistare on-line selezionando gli oggetti da mettere in un “carrello della spesa”
virtuale.
Deve essere possibile visualizzare il contenuto del carrello, modificare il contenuto del carrello, togliendo
alcuni articoli.
Al termine dell’acquisto va gestito il pagamento, che pu\'o avvenire con diverse modalit\'a.
Il sistema supporta differenti ricerche: per genere, per titolare del CD o DVD, per musicista partecipante,
per prezzo. Coerentemente, differenti modalit\'a di visualizzazione, sono altres\'i supportate.
Ogni vendita viene registrata indicando il cliente che ha acquistato, i prodotti acquistati, il prezzo
complessivo, la data di acquisto, l’ora, l’indirizzo IP del PC da cui \'e stato effettuato l’acquisto, la modalit\'a
di pagamento (bonifico, carta di credito, paypal) e la modalit\'a di consegna (corriere, posta, ...).
Per ogni cliente il sistema registra: il suo codice fiscale, il nome utente (univoco) con cui si \'e registrato,
la sua password, il nome, il cognome, la citt\'a di residenza, il numero di telefono ed eventualmente il
numero di cellulare.
Per i clienti autenticati, il sistema propone pagine specializzate che mostrano suggerimenti basati sul
genere dei precedenti prodotti acquistati.
Se il cliente ha fatto gi\'a 3 acquisti superiori ai 250 euro l’uno entro l’anno, il sistema gli propone sconti e
consegna senza spese di spedizione.
Il personale autorizzato del negozio pu\'o inserire tutti i dati dei CD e DVD in vendita. Il personale
inserisce anche il numero di pezzi a magazzino. Il sistema tiene aggiornato il numero dei pezzi a
magazzino durante la vendita e avvisa il personale del negozio quando un articolo (CD o DVD) scende
sotto i 2 pezzi presenti in magazzino.

\chapter{Idee di progettazione}
\section{Generale}
Abbiamo progettato un'applicazione Java che si interfaccia ad un server Postgres per l'acquisizione e
l'immagazzinamento dei dai. L'applicazione prevede un'interfaccia che permette di soddisfare le richieste
di due categorie di utenti:
\begin{itemize}
\item CLIENTI: i quali avranno la possibilit\'a di acquistare CD/DVD dal negozio online
\item PERSONALE: il quale avr\'a la possibili\'a di gestire il magazzino modificando dati nel database attraverso un GUI
\end{itemize}

\section{Versione attuale}
In questa versione dell'applicazione ci siamo concentrati nello sviluppo dell'interfaccia lato cliente.
L'utente avr\'a la possibilit\'a di sfogliare il catalogo con i prodotti, ottenere informazioni dettagliate per un prodotto
specifico, connettersi con un proprio account personale per effettuare acquisti, potendo modificare il contenuto 
del proprio carrello virtuale e visualizzare gli ordini gi\'a effettuati con tale account. Inoltre sar\'a possibilie creare un nuovo account qualora l'utente non dovesse disporne di uno.

\subsection{Database -  PostGres}
\subsection{GUI -  Java}

\chapter{Metodo di sviluppo}
La versione attuale dell'applicazione \'e stata implementata usando un metodo di progettazione agile, cercando similitudini e 
facendo confronti con i pi\'u famosi siti di e-commerce. Abbiamo organizzato il lavoro in modo da essere sempre efficenti utilizzando
un sito per la condivisione software che tiene traccia di tutte le modifiche effettuate, chiamato GitHub.\\
Grazie a questa piattaforma abbiamo potuto alternare scrittura di codice da casa ad incontri regolari in universit\'a per organizzare
le idee, confrontarsi, dividersi i compiti e darci delle scadenze ai lavori assegnatici.\\
Con GitHub abbiamo potuto tenere ben separati modelli sperimentali e non ancora testati, scritti e documentati su un branch developer, e i modelli che hanno superato la fase di testing, sul branch master.\\
In questo modo ci \'e sempre stato possibile fare un confronto per l'aggiornamento della nuova versione dell'applicazione,
e nel caso ne avessimo avuto necessit\'a la possibilit\'a di avere sempre una versione affidabile di partenza per un eventuale rollback.

\chapter{Diagrammi}
\section{Use Case Diagram}
\subsection{Use Case n.1}
\begin{itemize}
\item Sull'applicazione del negozio \'e illustrato il catalogo dei prodotti in vendita.
\item Cliccando sul nome dello strumento la finestra permette la visualizzazione dei dettagli del CD/DVD selezionato.
\item L'applicazione supporta diverse ricerche: genere, titolare del CD o DVD, musicista partecipante, prezzo.
\end{itemize}

Inserire USE CASE 1

\subsection{Use Case n.2}
\begin{itemize}
\item I clienti effettuare degli acquisti selezionando gli oggetti da mettere in un carrello virtuale.
\item \'E possibile visualizzare il contenuto del carrello, modificandone il contenuto, aggiungendo o togliendo alcuni articoli.
\item Al termine dell'acquisto viene gestito il pagamento, che pu\'o avvenire attraverso tre diverse modalit\'a.
\end{itemize}

Inserire USE CASE 2

\subsection{Use Case n.3}
\begin{itemize}
\item Tutti i clienti possono registrarsi al sito per effettuare acquisti.
\item I clienti autenticati ricevono pagine specializzate con prodotti consigliati, rispetto agli oggetti precedentemente acquistati.
\item Se il cliente ha gi\'a fatto tre acquisti superiori ai 250 euro l'un entro l'anno, il sistema propone ulteriori sconti e consegna senza spese di spezione.
\end{itemize}

Inserire USE CASE 3

\subsection{Use Case n.4}
\begin{itemize}
\item Il personale pu\'o inserire i dati per nuovi CD o DVD da mettere in vendita.
\item Il personale pu\'o aggiornare il numero di pezzi in magazzino per i CD/DVD in vendita.
\item Il personale viene avvisato se un oggetto del catalogo scende sotto i 2 pezzi in magazzino.
\end{itemize}

Inserire USE CASE 4

\section{Sequence Diagram}
\subsection{Sequence Diagram n.1}
\subsection{Sequence Diagram n.2}
\subsection{Sequence Diagram n.3}
\subsection{Sequence Diagram n.4}
\section{Activity Diagram}
\section{Class Diagram}

\newpage
\thispagestyle{empty}

\bibliography{biblio}
\bibliographystyle{plain}

\nocite{ingsft}
\nocite{ingreq}
\nocite{inglab}

\end{document}